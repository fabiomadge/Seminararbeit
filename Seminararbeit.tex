\documentclass[12pt, a4paper, draft]{report}
\usepackage[utf8]{inputenc}
\usepackage{setspace}
\usepackage[left=2.5cm, right=2.5cm]{geometry}
\usepackage[pdftex,bookmarks=true]{hyperref}
\usepackage{ngerman}
\usepackage{cite}
\onehalfspacing
\title{Kegelschnitte}
\author{Fabio Carlos Madge Pimentel\\
  Josef-Hofmiller-Gymnasium,\\
  Freising\\
  \texttt{fabio@madge.me}}
\date{November 2013}

\begin{document}

\begin{titlepage}
\begin{center}


\begin{minipage}{1\textwidth}
Josef-Hofmiller-Gymnasium Freising\\
Abiturjahrgang 2012/2014
\end{minipage}

\vspace{1cm}

\textbf{\uppercase{\Large Seminararbeit}}

\vspace{1cm}

Rahmenthema des Wissenschaftspropädeutischen Seminars:

xxx

Leitfach: Mathematik

\vspace{1cm}

{\large Thema der Arbeit:\\}
\vspace{.1cm}
\texttt{\textbf{\LARGE Kegelschnitte}}

\vfill

\begin{minipage}{0.4\textwidth}
\begin{flushleft}
Verfasser:\\
Fabio Carlos Madge Pimentel
\end{flushleft}
\end{minipage}
\begin{minipage}{0.4\textwidth}
\begin{flushright}
Kursleiterin:\\
OStRin Lohs
\end{flushright}
\end{minipage}

\vspace{0.6cm}

\begin{minipage}{0.4\textwidth}
\begin{flushleft}
Abgabetermin:
\end{flushleft}
\end{minipage}
\begin{minipage}{0.4\textwidth}
\begin{flushright}
12. November 2013
\end{flushright}
\end{minipage}

\vspace{1cm}

\begin{tabular}{|l|l|l|l|l|l|}
	\hline
	\textbf{Bewertung} & Note & Notenstufe in Worten & Punkte & & Punkte \\ \hline
	schriftliche Arbeit & & & & x 3 & \\ \hline
	Abschlusspräsentation & & & & x 1 & \\ \hline 
	\multicolumn{5}{ r| }{Summe} & \\ \cline{6-6}
	\multicolumn{5}{ r| }{Gesamtleistung nach § 61 (7) GSO = Summe/2 (gerundet)} & \\ \cline{6-6}
\end{tabular}

\vspace{2cm}

\hrule
\end{center}
Datum und Unterschrift der Kursleiterin bzw. des Kursleiters
\end{titlepage}

\setcounter{page}{2}
\tableofcontents
\clearpage

\chapter{Einführung}
	\setcounter{section}{0}
	\section{Einleitung}
	\section{Übersicht}
	\section{Begriffsklärung}
	\section{Dandelinsche Kugeln}

\chapter{Ellipse}
	\setcounter{section}{0}
	\section{Eigenschaften}
		Ellipsen entstanden aus Kreisen durch axiale Streckung. ~\cite{Pensel:1993uq}
	\section{Bild}
	\section{Herleitung}
	\section{Spezialfall: Kreis}

\chapter{Parabel}
	\setcounter{section}{0}
	\section{Eigenschaften}
	\section{Bild}
	\section{Herleitung}

\chapter{Hyperbel}
	\setcounter{section}{0}
	\section{Eigenschaften}
	\section{Bild}
	\section{Herleitung}

\bibliographystyle{plaindin}
\bibliography{cites}

\clearpage
\noindent Ich habe diese Seminararbeit ohne fremde Hilfe angefertigt und nur die im Literaturverzeichnis angeführten Quellen und Hilfsmittel benützt.

\vspace{2cm}

Freising, den \today

\end{document}