\setcounter{section}{0}
In diesem Kapitel sollen die Mittelpunktsgleichungen für die Kegelschnitte erarbeitet werden. Anhand derer können die kartesisches Koordinaten beliebiger Kegelschnitte ermittelt werden.\\
Der Name bezieht sich auf die Tatsache, dass der Mittelpunkt des Kegelschnitts mit dem Mittelpunkt des Koordinatensystem übereinstimmt. Bezüglich der Orientierung wird die 1. Hauptlage verwendet, welche die Hauptachsen der Kegelschnitte auf die x-Achse legt und F1 in den positiven x-Bereich.\\
Um die Aufgabe zu lösen wird Kenntnis um den Begriff der Exzentrizität vorausgesetzt, welche im folgenden Einschub aufgefrischt wird.
\section{Exzentrität}
Wie der Name suggeriert beschreibt die Exzentrizität die Abweichung eines Kegelschnitts von der Idealform eines Kreises. Sie wird in 2 Formen dargestellt. Zum einen die lineare, zum anderen die numerische Exzentrizität.
\subsection{Lineare Exzentrizität}
Die lineare Exzentrizität e gibt die Länge einer konkreten Strecke an. Der Abstand von Brennpunkt und Mittelpunkt wird durch sie ausgedrückt. Bestimmt wird e wie folgt:
\begin{displaymath}
	\begin{array}{rcl}
		e & = & \sqrt{a^2\pm b^2}
	\end{array}
\end{displaymath}
a entspricht dabei der bereits vorgestellten großen Halbachse. Die kleine Halbachse, die Strecke zwischen M und einem Nebenscheitel, wird als b bezeichnet.\\
Dass der Ausdruck für eine Ellipse eine Differenz ist lässt sich leicht zeigen. Für sie wird einer der Nebenscheitel S3 betrachtet. Dabei entsteht ein rechtwinkliges Dreieck $\triangle$ MF1S3. Mit Hilfe des Satzes des Pythagoras lässt sich also schreiben:
\begin{displaymath}
	\begin{array}{ccccl}
		\overline{M{S}_1}^2 & + & \overline{{S}_1{F'}_1}^2 & = & \overline{M{F'}_1}^2\\
		a^2 & + & b^2 & = & e^2\\
		&& e & = & \sqrt{a^2+b^2}
	\end{array}
\end{displaymath}
Die Substitution $(F1,S3)^2 = a^2$ folgt aus der Hyperbeldefinition.\\
Um zu zeigen, dass der Ausdruck für einen Hyperbel eine Summe ist, muss F1 transformiert werden. F1 wird bei konstanten Radius um M rotiert, bis es auf der Lot auf der Hauptachse durch S1 ist. Dadurch entsteht das rechtwinklige Dreieck $\triangle$ MS1F1'.
\begin{displaymath}
	\begin{array}{ccccl}
		\overline{M{F}_1}^2 & + & \overline{M{S}_3}^2 & = & \overline{{F}_1{S}_3}^2\\
		e^2 & + & b^2 & = & a^2\\
		&& e & = & \sqrt{a^2-b^2}
	\end{array}
\end{displaymath}
Da die Parabel über keinen sinnvollen Mittelpunkt verfügt, ist die Betrachtung in diesem Fall unnötig.\\
\subsection{Numerische Exzentrizität}
Die numerische Exzentrizität $\epsilon$ drückt das Verhältnis von e zu a aus. So trifft somit eine Aussage darüber, ob der Brennpunkt kürzer, oder weiter vom Mittelpunkt entfernt ist als der Hauptscheitel.
\begin{displaymath}
	\epsilon = \frac{e}{a}
\end{displaymath}
Anhand der numerische Exzentrizität kann auch die Ausprägung eines Kegelschnitts untersucht werden 
\begin{displaymath}
   \text{Ausprägung($\epsilon$)} = \left\{
     \begin{array}{lr}
       Kreis & \epsilon = 0 \\
       Ellipse & \epsilon < 1\\
       Parabel & \epsilon = 1\\
       Hyperbel & \epsilon > 1
     \end{array}
   \right.
\end{displaymath}
\section{Ellipse}
Ein weiteres Mal wird die Definition als Basis der Herleitung genutzt. Dabei werden die Strecken durch bekannte Größen ersetzt.
\begin{displaymath}
	\begin{array}{rcl}
		\overline{{PF}_1} + \overline{{PF}_2}  & = & 2a\\
		\sqrt{{(e+x)}^2 + y^2} + \sqrt{{(e-x)}^2 + y^2} & = & 2a\\
		\sqrt{{(e+x)}^2 + y^2} & = & 2a - \sqrt{{(e-x)}^2 + y^2}\\
		\sqrt{{(e+x)}^2 + y^2}^2 & = & \left (2a - \sqrt{{(e-x)}^2 + y^2}\right )^2\\
		e^2 + 2ex + x^2 + y^2 & = & 4a^2 - 4a \sqrt{(e-x)^2+y^2} + e^2 - 2ex + x^2 + y^2\\
		a\sqrt{{(e-x)}^2 + y^2} & = & a^2 - ex\\
		\left (a\sqrt{{(e-x)}^2 + y^2}\right )^2 & = & (a^2 - ex)^2\\
		a^2e^2 - 2a^2ex + a^2x^2 + a^2y^2 &=& a^4 - 2a^2ex + e^2x^2\\
		a^2x^2 - e^2x^2 + a^2y^2 &=& a^4 - a^2e^2\\
		\left(a^2 - e^2\right)x^2 + a^2y^2 &=& a^2\left(a^2 - e^2\right)\\
		b^2x^2 + a^2y^2 &=& a^2b^2\\
		\frac{x^2}{a^2} + \frac{y^2}{b^2} &=& 1
	\end{array}
\end{displaymath}
Die Substitution $\left(a^2 - e^2\right) = b^2$ folgt aus der Definition der linearen Exzentrizität. Wird der Spezialfall des Kreises betrachtet kann die Gleichung noch deutlich vereinfacht werden.
\begin{displaymath}
	\begin{array}{rcl}
		\frac{x^2}{r^2} + \frac{y^2}{r^2} &=& 1\\
		x^2 + y^2 &=& r^2
	\end{array}
\end{displaymath}
Da sowohl a, als auch b gleich r sind können bei ersetzt werden. 
\section{Parabel}
Auch bei der Parabel wird die wird die Definition zur Herleitung herangezogen. Jedoch muss der Parabelfaktor p eingeführt werden, welcher den Abstand zwischen F und F' angibt. Nun können die Strecken ersetzt werden.
\begin{displaymath}
	\begin{array}{rcl}
		\overline{PF} &=& \overline{Pl}\\
		\sqrt{\left(x-\frac{p}{2}\right)^2+y^2} &=& \frac{p}{2} + x\\
		\left(x-\frac{p}{2}\right)^2+y^2 &=& \left(\frac{p}{2} + x\right)^2\\
		x^2 - px + \frac{p^2}{4} + y^2 &=& x^2 + px + \frac{p^2}{4}\\
		y^2 &=& 2px
	\end{array}
\end{displaymath}
\section{Hyperbel}
Bei der Herleitung für die Hyperbel muss eine Fallunterscheidung eingeführt werden, um sowohl den x-positiven als auch den y-negativen Ast auf die Definition zurückzuführen. Für den rechten Ast wird wie folgt vorgegangen:
\begin{displaymath}
	\begin{array}{rcl}
		\overline{{PF}_1} - \overline{{PF}_2}  & = & 2a\\
		\sqrt{{(e+x)}^2 + y^2} - \sqrt{{(e-x)}^2 + y^2} & = & 2a\\
		\sqrt{{(e+x)}^2 + y^2} & = & 2a + \sqrt{{(e-x)}^2 + y^2}\\
		\sqrt{{(e+x)}^2 + y^2}^2 & = & \left (2a + \sqrt{{(e-x)}^2 + y^2}\right )^2\\
		e^2 + 2ex + x^2 + y^2 & = & 4a^2 + 4a \sqrt{(e-x)^2+y^2} + e^2 - 2ex + x^2 + y^2\\
		a\sqrt{{(e-x)}^2 + y^2} & = & ex - a^2\\
		\left (a\sqrt{{(e-x)}^2 + y^2}\right )^2 & = & (ex - a^2)^2\\
		a^2e^2 - 2a^2ex + a^2x^2 + a^2y^2 &=& e^2x^2 - 2a^2ex + a^4\\
		a^2e^2 - a^4 &=& e^2x^2 - a^2x^2 - a^2y^2\\
		e^2x^2 - a^2x^2 - a^2y^2 &=& a^2e^2 - a^4\\
		\left(e^2 - a^2\right)x^2 - a^2y^2 &=& a^2\left(e^2 - a^2\right)\\
		b^2x^2 - a^2y^2 &=& a^2b^2\\
		\frac{x^2}{a^2} - \frac{y^2}{b^2} &=& 1
	\end{array}
\end{displaymath}
Auch in diesem Fall kann die Substitution $\left(e^2 - a^2\right) = b^2$ auf die Definition der linearen Exzentrizität zurückgeführt werden. Für den 2. Fall wird lediglich die ausgehende Substitution leicht modifiziert:
\begin{displaymath}
	\begin{array}{rcl}
		\overline{{PF}_1} - \overline{{PF}_2}  & = & 2a\\
		\sqrt{{(e+(-x))}^2 + y^2} - \sqrt{{((-x)-e)}^2 + y^2} & = & 2a\\
		\sqrt{{(e-x)}^2 + y^2} & = & 2a + \sqrt{{(-e-x)}^2 + y^2}\\
		\sqrt{{(e-x)}^2 + y^2}^2 & = & \left (2a + \sqrt{{(-e-x)}^2 + y^2}\right )^2\\
		e^2 - 2ex + x^2 + y^2 & = & 4a^2 + 4a \sqrt{(-e-x)^2+y^2} + e^2 + 2ex + x^2 + y^2\\
		a\sqrt{{(-e-x)}^2 + y^2} & = & -ex - a^2\\
		\left (a\sqrt{{(-e-x)}^2 + y^2}\right )^2 & = & (-ex - a^2)^2\\
		a^2e^2 + 2a^2ex + a^2x^2 + a^2y^2 &=& e^2x^2 + 2a^2ex + a^4\\
		a^2e^2 - a^4 &=& e^2x^2 - a^2x^2 - a^2y^2\\
		e^2x^2 - a^2x^2 - a^2y^2 &=& a^2e^2 - a^4\\
		\left(e^2 - a^2\right)x^2 - a^2y^2 &=& a^2\left(e^2 - a^2\right)\\
		b^2x^2 - a^2y^2 &=& a^2b^2\\
		\frac{x^2}{a^2} - \frac{y^2}{b^2} &=& 1
	\end{array}
\end{displaymath}	