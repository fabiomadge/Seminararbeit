Die Kegelschnitte sind ein sehr altes Thema der Mathematik. Die ersten Gedanken über die Gruppe im Ganzen wurden wohl durch die alten Griechen gefasst. Wenig verwunderlich ist da die Tatsache, dass die älteste Veröffentlichung zu dem Thema aus eben diesem Kulturkreis stammt. Apollonios von Perge~\cite{Perge:1967} ebnete durch sein Werk den Weg für 2200 Jahre Erkenntnisse auf diesem Gebiet.\\
Dies blieb nicht ohne Folgen, wurde doch eine beeindruckende Menge an Erkenntnissen angehäuft. Gebrauch davon wird heute in den verschiedensten Disziplinen gemacht, etwa in der Astrophysik~\cite[3.8]{Jennings:1994} oder der Optik~\cite[3.5]{Jennings:1994}.\\
Jedoch stelle ich an das Folgende nicht den Anspruch der Vollständigkeit. So möchte ich im folgenden keinen Leitfaden zum Beherrschen der Kegelschnitte bieten, sondern viel mehr einen groben Überblick über dieses Gebiet schaffen. Dadurch soll ein leichter Zugang zu dem Thema geboten werden, welcher bei Bedarf punktuell ausgeweitet werden kann.\\
Letztendlich erhoffe ich mir für jeden Leser am Ende der Lektüre die Kenntnis um die Vier Kegelschnitte und deren enge Verwandtschaft. Über die Erkenntnis, die Zeit sinnvoll genutzt zu haben, würde ich mich selbstverständlich besonders freuen.\\