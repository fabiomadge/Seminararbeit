\setcounter{section}{0}
\section{Eigenschaften}
Schnitt von Doppelkegel K und Ebene E im Raum
Gruppierung nach dem Verhältnis von Ebene und Mantellinie V = \angle(E, Grundfläche(K)) - \angle(Mantellinie(K), Grundfläche(K)):
1. Ellipse V < 0 [Spezialfall: Kreis (E||Grundfläche(K))]
2. Parabel V = 0
3. Hyperbel V > 0
Blödsinn, wenn Schnittpunkt (0,0)
\section{Bild}
Ellipsen entstanden aus Kreisen durch axiale Streckung. ~\cite{Bourke:1988}
\section{Herleitung}