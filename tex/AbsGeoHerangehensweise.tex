\setcounter{section}{0}
\section{Modell}
Die im Folgenden beschriebene Herangehensweise~\cite[S.83f]{Jennings:1994} macht dank ihrer Reduziertheit die Zusammenhänge der einzelnen Kegelschnittsausprägungen deutlich und bietet einen niedrigschwelligen Einstieg in die Gruppe der Kegelschnitte.\\
Es wird ein Doppelkegel $K$ definiert\cite[S.793]{Bronstein:1996}. Dieser kann als Rotationsköper einer Geraden der Form $y=mx$ mit der $y$-Achse als Rotationsachse beschrieben werden. Des weiteren wird eine beliebige Ebene $E$ definiert. Einzig Ebenen, welche durch das Zentrum des Kegels, den Ursprung des Koordinatensystems, durchlaufen, werden ausgeschlossen, nachdem die Schnittfigur keinen sinnvollen Kegelschnitt ergibt.
\begin{displaymath}
	\begin{array}{lrl}
		\text{$K$: }x^2+y^2=\frac{z^2}{m^2} & \{x,y,z\in \mathbb{R};m\in \mathbb{R}\}\\
		\text{$E$: }0=ax+by+cz+t & \{x,y,z\in \mathbb{R}; a,b,c,t\in \mathbb{R}|(a \lor b \lor c) \neq 0\} & (0,0,0) \notin \text{E}
	\end{array}
\end{displaymath}
Um nun eine Fallunterscheidung durchzuführen werden die Winkel \(\alpha\) und \(\beta\) eingeführt und deren Verhältnis $V$. $V$ beschreibt das Verhältnis der Ebene $E$ zur Mantellinie von $K$ durch die Differenz von \(\alpha\) und \(\beta\).\\
Der Winkel \(\alpha\) wird zwischen der Ebene $E$ und der Grundfläche $g$ des Kegels K aufgespannt. \(\beta\) wird zwischen der Mantellinie von $K$ und der Grundfläche $g$ des Kegels $K$ gemessen.\\

\section{Befunde}
Anhand des eben aufgestellten Verhältnisses $V$ können die Kegelschnitte nun gruppiert werden.\\
Im ersten Fall ist die Mantellinie von $K$ steiler als $E$, wodurch die Ebene $K$ nur einmal schneidet. $V$ befindet sich folglich im positiven Bereich. Die dabei entstehende, endlich große, Fläche wird als Ellipse bezeichnet. Einen Spezialfall der Ellipse stellt der Kreis dar, welcher entsteht, wenn $E$ parallel zur Grundfläche des Kegels steht.\\
Ist es jedoch umgekehrt und die Mantellinie von $K$ ist flacher als $E$, bietet sich ein komplett anderes Bild. Es entstehen nun zwei entgegengesetzte Flächen, welche vom gemeinsamen Mittelpunkt abgewandt nicht begrenzt sind. Diese Schnittebene wird als Hyperbel bezeichnet.\\
Dazwischen befindet sich der Parallelfall, wobei die Mantellinie von $K$ und $E$ parallel zueinander sind. Dabei entsteht jedoch nur eine Fläche, welche aber auch offen ist. Man nennt diesen Fall Parabel.\\
\begin{displaymath}
   \text{Ausprägung($v$)} = \left\{
     \begin{array}{lr}
       Kreis & E \parallel \text{Grundfläche($K$)} \wedge v < 0 \\
       Ellipse & v < 0\\
       Parabel & v = 0\\
       Hyperbel & v > 0
     \end{array}
   \right.
\end{displaymath}